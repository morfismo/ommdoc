\documentclass[onecolumn,11pt,final]{ommdoc}
\title{Problemas de conteo}
\author{Didier Solís}
\date{4 de junio de 2013}
\begin{document}
\maketitle
\thispagestyle{empty}
\nonzeroparskip
\vspace{1cm}

\problema 	En el juego de Melate, para seleccionar la combinación ganadora se escogen 6 números entre $1$ y $44$. ¿Cuántas combinaciones ganadoras tienen una pareja de números consecutivos?

\problema En una alcancía hay $2n$ monedas de peso, $2n$ monedas de cinco pesos y $2n$ monedas de 10 pesos. Luis y Jhoni rompen la alcancía y se reparten las monedas de tal modo que a cada uno le toca el mismo número de monedas. ¿De cuántas formas puede hacerse esto?  {\small Nota: Importa la distribución de monedas, no el monto total}

\problema En la lengua \emph{marinólica} hay 2012 letras y todas las plabras están escritas
de manera cada letra antecede en el orden lexicográfico a todas las letras
a su derecha. ¿Cuántas palabras de 10 letras tiene la lengua marinólica?

\problema Encuentra el número de palabras de longitud 2013 compuestas con las
letras A y B, de manera que la "sílaba" AB aparezca exactamente 700 veces.

\problema Una \emph{composición} de un entero positivo $n$ es una expresión de $n$ como suma
de dos o más enteros positivos. Por ejemplo 
\[3 + 1, 1 + 3, 2 + 2, 2 + 1 + 1,
1+2+1, 1+1+2 \text{ y } 1+1+1+1\]
son todas las composiciones de 4. Halla el número $c(n,k)$ de todas las composiciones del entero $n$ en $k$ sumandos.

\problema De cuantas formas pueden escogerse $n$ enteros $a_1,a_2,\ldots,a_n$ tales que 
\[
1 \le a_1\le a_2 \le a_3 \le \cdots \le a_n \le n \text{?}
\]
\end{document}
